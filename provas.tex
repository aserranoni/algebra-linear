\documentclass[11pt,a4paper]{article}
\usepackage{estilosexercicios}

% ---------------------------------------------------
\title{Álgebra Linear}
\author{MAT5730}
\date{2 semestre de 2019}

\begin{document}
\maketitle
\tableofcontents
\newpage
\begin{comment}

\begin{center}
\large\textbf{\textcolor{Floresta}{Provas}}\\
\end{center}

\end{comment}

\section{\textcolor{Floresta}{Prova 1}}

\begin{exercicio} Sejam $a,b,c,d \in \mathbb{R}.$ Encontre o valor de
\[
\det \begin{bmatrix}
a & b & c & d\\
9 & 8 & 7 & 6\\
1 & 1 & 1 & 1\\
2020 & 2018 & 2017 & 2016
\end{bmatrix}
\]
\end{exercicio}
\solucao{
Como o determinante é uma forma $3$-linear nas linhas da matriz, podemos notar que
\[
\det \begin{bmatrix}
a & b & c & d\\
9 & 8 & 7 & 6\\
1 & 1 & 1 & 1\\
2020 & 2018 & 2017 & 2016
\end{bmatrix} = \det \begin{bmatrix}
a & b & c & d\\
6 & 6 & 6 & 6\\
1 & 1 & 1 & 1\\
2016 & 2016 & 2016 & 2016
\end{bmatrix}
 + \det \begin{bmatrix}
a & b & c & d\\
3 & 2 & 1 & 0\\
1 & 1 & 1 & 1\\
4 & 2 & 1 & 0
\end{bmatrix}
\]
O determinante da primeira matriz é nulo, pois tem três linhas com valores iguais. Já o determinante da segunda matriz pode ser facilmente calculado utilizando o Teorema de Laplace:
\[\det \begin{bmatrix}
a & b & c & d\\
3 & 2 & 1 & 0\\
1 & 1 & 1 & 1\\
4 & 2 & 1 & 0
\end{bmatrix} = (-1)^{1+1} a \cdot \textcolor{Green}{\det \begin{bmatrix}
2 & 1 & 0\\
1 & 1 & 1\\
2 & 1 & 0
\end{bmatrix}} + (-1)^{2+1} b \cdot \textcolor{Blue}{\det
\begin{bmatrix}
3  & 1 & 0\\
1  & 1 & 1\\
4  & 1 & 0
\end{bmatrix}} + \]\[(-1)^{3+1} c \cdot \textcolor{Red}{\det \begin{bmatrix}
3 & 2 & 0\\
1 & 1 & 1\\
4 & 2 & 0
\end{bmatrix}} + (-1)^{4+1} d \cdot \textcolor{Laranja}{\det \begin{bmatrix}
3 & 2 & 1 \\
1 & 1 & 1 \\
4 & 2 & 1 \\
\end{bmatrix}} =  (-1)^{1+1} a \cdot \textcolor{Green}{\det \begin{bmatrix}
2 & 1 & 0\\
1 & 1 & 1\\
2 & 1 & 0
\end{bmatrix}} + (-1)^{2+1} b \cdot \textcolor{Blue}{\det
\begin{bmatrix}
3  & 1 & 0\\
1  & 1 & 1\\
4  & 1 & 0
\end{bmatrix}} + \]\[(-1)^{3+1} c \cdot \textcolor{Red}{\det \begin{bmatrix}
3 & 2 & 0\\
1 & 1 & 1\\
4 & 2 & 0
\end{bmatrix}} + (-1)^{4+1} d \cdot \textcolor{Laranja}{\det \begin{bmatrix}
3 & 2 & 1 \\
1 & 1 & 1 \\
4 & 2 & 1 \\
\end{bmatrix}}
\]
}

\begin{exercicio}
 Seja $V$ um $K$-espaço vetorial e $T \in \mathcal{L}(V).$ Seja $W \subseteq V$ um subespaço $T$-invariante de $V.$
\dividiritens{
\task[\pers{a}] Mostre que, se $T$ é diagonalizável, então a restrição $T \upharpoonleft_W$ é diagonalizável.
\task[\pers{b}] Seja $\mbox{Spec } T = \{ \lambda_1, \ldots, \lambda_n \}$ o conjunto de autovalores de $T,$ onde $\lambda_i \neq \lambda_j$ para $i \neq j.$ Quantos subespaços $T$-invariantes o espaço vetorial $V$ possui?
}
\end{exercicio}
\solucao{}
\begin{exercicio}
Encontre o polinômio característico e o polinômio minimal da matriz
\[
A = \begin{bmatrix}
1 & -1 & 1 & -1 & 1 & -1 \\
1 & -1 & 1 & -1 & 1 & -1 \\
1 & -1 & 1 & -1 & 1 & -1 \\
1 & -1 & 1 & -1 & 1 & -1 \\
1 & -1 & 1 & -1 & 1 & -1 \\
1 & -1 & 1 & -1 & 1 & -1
\end{bmatrix} \in \mathcal{M}_6(K)
\]
\end{exercicio}
\solucao{
Temos que
\[
p_A(\lambda) = \det(A - \lambda I) = \det \begin{bmatrix}
1-\lambda & -1 & 1 & -1 & 1 & -1 \\
1 & -1-\lambda & 1 & -1 & 1 & -1 \\
1 & -1 & 1-\lambda & -1 & 1 & -1 \\
1 & -1 & 1 & -1-\lambda & 1 & -1 \\
1 & -1 & 1 & -1 & 1-\lambda & -1 \\
1 & -1 & 1 & -1 & 1 & -1-\lambda
\end{bmatrix} =  \]\[ \det \begin{bmatrix}
-\lambda & -1 & 1 & -1 & 1 & -1 \\
- \lambda & -1-\lambda & 1 & -1 & 1 & -1 \\
0 & -1 & 1-\lambda & -1 & 1 & -1 \\
0 & -1 & 1 & -1-\lambda & 1 & -1 \\
0 & -1 & 1 & -1 & 1-\lambda & -1 \\
0 & -1 & 1 & -1 & 1 & -1-\lambda
\end{bmatrix} = \]\[  -\lambda \det \begin{bmatrix}
-1-\lambda & 1 & -1 & 1 & -1 \\
-1 & 1-\lambda & -1 & 1 & -1 \\
-1 & 1 & -1-\lambda & 1 & -1 \\
 -1 & 1 & -1 & 1-\lambda & -1 \\
 -1 & 1 & -1 & 1 & -1-\lambda
\end{bmatrix} + \lambda \det 
\begin{bmatrix}
-1 & 1 & -1 & 1 & -1 \\
-1 & 1-\lambda & -1 & 1 & -1 \\
-1 & 1 & -1-\lambda & 1 & -1 \\
-1 & 1 & -1 & 1-\lambda & -1 \\
-1 & 1 & -1 & 1 & -1-\lambda
\end{bmatrix}.
\]
Temos também que
\[
\det \begin{bmatrix}
-1-\lambda & 1 & -1 & 1 & -1 \\
-1 & 1-\lambda & -1 & 1 & -1 \\
-1 & 1 & -1-\lambda & 1 & -1 \\
 -1 & 1 & -1 & 1-\lambda & -1 \\
 -1 & 1 & -1 & 1 & -1-\lambda
\end{bmatrix} = \det \begin{bmatrix}
-\lambda & 1 & -1 & 1 & -1 \\
-\lambda & 1-\lambda & -1 & 1 & -1 \\
0 & 1 & -1-\lambda & 1 & -1 \\
0 & 1 & -1 & 1-\lambda & -1 \\
0 & 1 & -1 & 1 & -1-\lambda
\end{bmatrix} = \]\[ - \lambda \det \begin{bmatrix}
 1-\lambda & -1 & 1 & -1 \\
 1 & -1-\lambda & 1 & -1 \\
 1 & -1 & 1-\lambda & -1 \\
 1 & -1 & 1 & -1-\lambda
\end{bmatrix}  + \lambda \det \begin{bmatrix}
1 & -1 & 1 & -1 \\
1 & -1-\lambda & 1 & -1 \\
1 & -1 & 1-\lambda & -1 \\
1 & -1 & 1 & -1-\lambda
\end{bmatrix} = \]\[
- \lambda \det \begin{bmatrix}
-\lambda & -1 & 1 & -1 \\
-\lambda & -1-\lambda & 1 & -1 \\
0 & -1 & 1-\lambda & -1 \\
0 & -1 & 1 & -1-\lambda
\end{bmatrix}  + \lambda \det \begin{bmatrix}
0 & -1 & 1 & -1 \\
-\lambda & -1-\lambda & 1 & -1 \\
0 & -1 & 1-\lambda & -1 \\
0 & -1 & 1 & -1-\lambda
\end{bmatrix} = \]\[ - \lambda \left( -\lambda
\det \begin{bmatrix}
 -1-\lambda & 1 & -1 \\
 -1 & 1-\lambda & -1 \\
 -1 & 1 & -1-\lambda
\end{bmatrix} + \lambda \begin{bmatrix}
-1 & 1 & -1 \\
-1 & 1-\lambda & -1 \\
-1 & 1 & -1-\lambda
\end{bmatrix} \right)
+ \lambda \left(  \lambda   \det \begin{bmatrix}
-1 & 1 & -1 \\
-1 & 1-\lambda & -1 \\
-1 & 1 & -1-\lambda
\end{bmatrix}  \right) = \]\[  - \lambda \left( -\lambda
(-\lambda^3 - \lambda^2) - \lambda^3 \right)
+ \lambda \left(  - \lambda^3 \right) = - \lambda^5 - \lambda^4
\]
e
\[
 \det 
\begin{bmatrix}
-1 & 1 & -1 & 1 & -1 \\
-1 & 1-\lambda & -1 & 1 & -1 \\
-1 & 1 & -1-\lambda & 1 & -1 \\
-1 & 1 & -1 & 1-\lambda & -1 \\
-1 & 1 & -1 & 1 & -1-\lambda
\end{bmatrix} =  \det 
\begin{bmatrix}
0 & 1 & -1 & 1 & -1 \\
-\lambda & 1-\lambda & -1 & 1 & -1 \\
0 & 1 & -1-\lambda & 1 & -1 \\
0 & 1 & -1 & 1-\lambda & -1 \\
0 & 1 & -1 & 1 & -1-\lambda
\end{bmatrix} = \]\[ \lambda \det 
\begin{bmatrix}
 1 & -1 & 1 & -1 \\
 1 & -1-\lambda & 1 & -1 \\
 1 & -1 & 1-\lambda & -1 \\
 1 & -1 & 1 & -1-\lambda
\end{bmatrix} = \]\[\lambda \det 
\begin{bmatrix}
0 & -1 & 1 & -1 \\
 -\lambda & -1-\lambda & 1 & -1 \\
0 & -1 & 1-\lambda & -1 \\
0 & -1 & 1 & -1-\lambda
\end{bmatrix} = \lambda \left(\lambda \det \begin{bmatrix}
 -1 & 1 & -1 \\
 -1 & 1-\lambda & -1 \\
 -1 & 1 & -1-\lambda
\end{bmatrix} \right) = - \lambda^4
\]
Assim, 
\[
\begin{bmatrix}
1 & -1 & 1 & -1 & 1 & -1 \\
1 & -1 & 1 & -1 & 1 & -1 \\
1 & -1 & 1 & -1 & 1 & -1 \\
1 & -1 & 1 & -1 & 1 & -1 \\
1 & -1 & 1 & -1 & 1 & -1 \\
1 & -1 & 1 & -1 & 1 & -1
\end{bmatrix} = -\lambda \det \begin{bmatrix}
-1-\lambda & 1 & -1 & 1 & -1 \\
-1 & 1-\lambda & -1 & 1 & -1 \\
-1 & 1 & -1-\lambda & 1 & -1 \\
 -1 & 1 & -1 & 1-\lambda & -1 \\
 -1 & 1 & -1 & 1 & -1-\lambda
\end{bmatrix} + \]\[ \lambda \det 
\begin{bmatrix}
-1 & 1 & -1 & 1 & -1 \\
-1 & 1-\lambda & -1 & 1 & -1 \\
-1 & 1 & -1-\lambda & 1 & -1 \\
-1 & 1 & -1 & 1-\lambda & -1 \\
-1 & 1 & -1 & 1 & -1-\lambda
\end{bmatrix} = -\lambda (- \lambda^5 - \lambda^4)  + \lambda(-\lambda^4) = \lambda^6 
\]
Assim, o polinômio característico de $A$ é $p_A(\lambda) = \lambda^6.$ Observe agora que $A^2 = 0.$ Assim, o polinômio minimal de $A$ é $m_A(\lambda) = \lambda^2.$
}
\begin{exercicio}
 Dentre as três matrizes abaixos, quais delas são semelhantes?
\[
A = \begin{pmatrix}
1 & 1 & 1 \\
-1& -1 & -1 \\
1 & 1 & 1
\end{pmatrix}  \quad B = \begin{pmatrix}
1 & 0 & 0 \\
0& 0 & 0 \\
0 & 0 &0
\end{pmatrix} \quad C = \begin{pmatrix}
-1 & -2 & -3 \\
2& 4 & 6 \\
-1& -2 &-3
\end{pmatrix}
\]
\end{exercicio}
\solucao{
Como as matrizes são $3 \times 3,$ duas matrizes são semelhantes se e somente se possuem o mesmo polinômio minimal e o mesmo polinômio característico. Calculemos então os polinômios minimal e característico de cada matriz:
\[
p_A(\lambda) = \det(A - \lambda I) = \det \begin{pmatrix}
1-\lambda & 1 & 1 \\
-1& -1-\lambda & -1 \\
1 & 1 & 1-\lambda
\end{pmatrix} = \lambda^3 - \lambda^2 = \lambda^2(\lambda - 1)
\]
O polinômio minimal será $m_A(\lambda) = \lambda(\lambda - 1).$
\[
p_B(\lambda) = \det(B - \lambda I) = \det \begin{pmatrix}
1-\lambda & 0 & 0 \\
0& -\lambda & 0 \\
0 & 0 &-\lambda
\end{pmatrix} = \lambda^2(\lambda - 1)
\]
O polinômio minimal será $m_B(\lambda) = \lambda(\lambda - 1).$
\[p_C(\lambda) = \det(C - \lambda I) = \det \begin{pmatrix}
-1- \lambda & -2 & -3 \\
2& 4- \lambda & 6 \\
-1& -2 &-3- \lambda
\end{pmatrix} = \lambda^3
\]
O polinômio minimal será $m_C(\lambda) = \lambda^2.$

Logo, temos que as matrizes $A$ e $B$ são semelhantes.
}
\begin{exercicio} Seja $V = \mathcal{P}_3(\mathbb{R})$ o espaço vetorial formado por todos os polinômios de grau menor ou igual a $3.$ Considere o operador $T \in \mathcal{L}(V)$ dado por
\[
T(f(x)) = (x-1) f^{\prime}(x)
\]
\dividiritens{
\task[\pers{a}] Determine a matriz do operador $T$ em $\mathcal{P}_3(\mathbb{R})$ em relação à base canônica desse espaço.
\task[\pers{b}] Encontre o polinômio característico e o polinômio minimal de $T.$
\task[\pers{c}] $T$ é diagonalizável? Em caso afirmativo, apresente uma base de autovetores para $V.$
}
\end{exercicio}
\solucao{
\dividiritens{
\task[\pers{a}] Considere a base canônica $B = \{1, x, x^2, x^3 \}$ de $V.$ Vamos encontrar a matriz que representa $T$ nessa base. Para tanto, note que
\[
T(1) = (x-1)(1)^{\prime} = 0, \quad T(x) = (x-1)(x)^{\prime} = x-1,
\]
\[
T(x^2) = (x-1)(x^2)^{\prime} = 2x^2 - 2x, \quad T(x^3) = (x-1)(x^3)^{\prime} = 3x^3 - 3x^2.
\]
Logo,
\[
[T]_B = \begin{pmatrix}
0  & 0 & 0 & 0 \\
-1 & 1 & 0 & 0 \\
0 & -2 & 2 & 0 \\
0 & 0 & -3 & 3
\end{pmatrix}
\]


\task[\pers{b}] %Para encontrar os autovalores e autovetores de $T,$ precisamos encontrar a matriz que representa $T$ 
Encontremos os autovalores de $[T]_B:$
\[
p_{[T]_B}(\lambda) = \det([T]_B - \lambda I) = \det \begin{pmatrix}
-\lambda & 0 & 0 & 0 \\
-1 & 1-\lambda & 0 & 0 \\
0 & -2 & 2-\lambda & 0 \\
0 & 0 & -3 & 3-\lambda
\end{pmatrix} = \lambda^4 - 6\lambda^3 + 11 \lambda^2 - 6 \lambda = \lambda (\lambda - 1)(\lambda - 2)( \lambda - 3).
\]
Assim, os autovalores são $0, 1, 2$ e $3.$ Agora, vejamos os autovetores:
\[
v = (a, b, c, d) \in \Ker([T]_B - 0 I) \Rightarrow \begin{pmatrix}
0  & 0 & 0 & 0 \\
-1 & 1 & 0 & 0 \\
0 & -2 & 2 & 0 \\
0 & 0 & -3 & 3
\end{pmatrix}\begin{pmatrix}
a \\
b \\
c \\
d
\end{pmatrix} = 0 \Rightarrow \begin{cases}
a + b = 0 \\
-2b + 2c = 0 \\
-3c + 3d = 0
\end{cases} \Rightarrow v = a(1,-1,-1,-1)
\]
}

}

\newpage
\section{\textcolor{Floresta}{Prova 2}}

\begin{exercicio}
 Suponha que $\mathbb{C}^2$ esteja munido de um produto interno $\prin{,}$ e que a base $\mathcal{B} = \{ (1,1), (1,i) \}$ seja ortonormal em relação a $\prin{,}.$
 \dividiritens{
 \task[\pers{a}] Calcule os seguintes produtos: $\prin{(i,1),(-i,i)}, \prin{(-1,1),(1,1)}$ e $\prin{(1,0),(0,1)}.$
  \task[\pers{b}] Seja $T \in \mathcal{L}(\mathbb{C}^2)$ um operador linear cuja matriz em relação à base canônica de $\mathbb{C}^2$ é
  \[
  [T]_{\mbox{can}} = \begin{pmatrix}
  i+1 & -1 \\
  i & 0
  \end{pmatrix}.
  \]
  
  Determine o operador adjunto $T^{*}$ (relativamente ao produto interno $\prin{,}$ acima) e verifique se $T$ é normal.
 }
\end{exercicio}
\solucao{
\dividiritens{
\task[\pers{a}] Sejam $e_1 = (1,1)$ e $e_2 = (1,i).$ Como $\mathcal{B}$ é ortonormal, então $\prin{e_i, e_j} = \delta_{ij}.$ Com base nisso, vamos calcular cada um dos produtos requisitados:
\begin{itemize}
    \item $\prin{(i,1),(-i,i)} = 0,$ pois
    
\[\prin{(i,1),(-i,i)} = \prin{(i+1)e_1 - e_2, -e_1 + (1-i)e_2} = \]\[ \prin{(i+1)e_1, -e_1} + \prin{(i+1)e_1, (1-i)e_2} + \prin{-e_2, -e_1} + \prin{-e_2, (1-i)e_2} = \]\[ (i+1)\overline{(-1)} \underbrace{\prin{e_1, e_1}}_{=1}+ (i+1)\overline{(1-i)} \underbrace{\prin{e_1, e_2}}_{=0} + (-1)\overline{(-1)} \underbrace{\prin{e_2, e_1}}_{=1} + (-1)\overline{1-i} \underbrace{\prin{e_2, e_2}}_{=1} = \]
\[
-(i-1) + (i-1) = 0
\]
%\[\begin{array}{rcl}\prin{(i,1),(-i,i)} &=& \prin{(i+1)e_1 - e_2, -e_1 + (1-i)e_2} \\ &=& \prin{(i+1)e_1, -e_1} + \prin{(i+1)e_1, (1-i)e_2} + \prin{-e_2, -e_1} + \prin{-e_2, (1-i)e_2} \\ &=&  (i+1)\overline{(-1)} \underbrace{\prin{e_1, e_1}}_{=1}+ (i+1)\overline{(1-i)} \underbrace{\prin{e_1, e_2}}_{=0} + (-1)\overline{(-1)} \underbrace{\prin{e_2, e_1}}_{=1} + (-1)\overline{1-i} \underbrace{\prin{e_2, e_2}}_{=1} \\ &=& \\\end{array}\]
    \item $\prin{(-1,1),(1,1)} = i,$ pois
    
\[\prin{(-1,1),(1,1)} = \prin{(ie_1 -(i+1)e_2, e_1} = \]\[ \prin{ie_1, e_1} + \prin{-(i+1)e_2, e_1} = \]\[ i \underbrace{\prin{e_1, e_1}}_{=1} -(i+1) \underbrace{\prin{e_2, e_1}}_{=0} = i\]

    \item $\prin{(1,0),(0,1)} = -1,$ pois
    
\[\prin{(-1,1),(1,1)} = \prin{\left(\frac{1 - i}{2}\right)e_1 + \left(\frac{1+i}{2} \right)e_2, \left(\frac{i - 1}{2}\right)e_1 - \left(\frac{1+i}{2} \right)e_2} = \]\[
\prin{\left(\frac{1 - i}{2}\right)e_1, \left(\frac{i - 1}{2}\right)e_1} + \prin{\left(\frac{1 - i}{2}\right)e_1, - \left(\frac{1+i}{2} \right)e_2} + \]\[ \prin{\left(\frac{1+i}{2} \right)e_2, \left(\frac{i - 1}{2}\right)e_1} + \prin{\left(\frac{1+i}{2} \right)e_2, - \left(\frac{1+i}{2} \right)e_2} = 
\]\[
\left(\frac{1 - i}{2}\right) \overline{\left(\frac{i - 1}{2}\right)}  \underbrace{\prin{e_1, e_1}}_{=1}+ \left(\frac{1 - i}{2}\right)\overline{  \left(-\frac{1+i}{2} \right) } \underbrace{\prin{e_1, e_2}}_{=0}+\]\[ \left(\frac{1+i}{2} \right) \overline{\left(\frac{i - 1}{2}\right)} \underbrace{\prin{e_2, e_1}}_{=0}+ \left(\frac{1+i}{2} \right) \overline{\left(-\frac{1+i}{2} \right)} \underbrace{\prin{e_2, e_2}}_{=1}= \]\[
= -\left(\frac{1 - i}{2}\right) \left(\frac{i + 1}{2}\right) + \left(\frac{1+i}{2} \right) \left(\frac{i-1}{2} \right) = -1 
\]
\end{itemize}
\task[\pers{b}] Dada a matriz apresentada, temos que 
\[
T(1,0) = \begin{pmatrix}
  i+1 & -1 \\
  i & 0
  \end{pmatrix}\begin{pmatrix}
  1 \\
  0
  \end{pmatrix} = \begin{pmatrix}
  i+1 \\
  i
  \end{pmatrix}
\]
\[
T(0,1) = \begin{pmatrix}
  i+1 & -1 \\
  i & 0
  \end{pmatrix}\begin{pmatrix}
  0 \\
  1
  \end{pmatrix} = \begin{pmatrix}
  -1 \\
  0
  \end{pmatrix}
\]
Do item anterior, sabemos que 
\[
(1,0) = \left(\frac{1 - i}{2}\right)e_1 + \left(\frac{1+i}{2} \right)e_2 \quad \mbox{e} \quad (0,1) =  \left(\frac{i - 1}{2}\right)e_1 - \left(\frac{1+i}{2} \right)e_2.
\]
Além disso, 
\[
(i+1,i) = \left(\frac{i+3}{2}\right)e_1 + \left(\frac{i-1}{2} \right)e_2 \quad \mbox{e} \quad (-1,0) =  \left(\frac{i - 1}{2}\right)e_1 - \left(\frac{1+i}{2} \right)e_2.
\]
}

}

\begin{exercicio}
Determine o polinômio de grau $2$ cujo gráfico melhor se ajusta aos pontos $(-1,-1), (0,0), (1,1)$ e $(-1,1).$
\end{exercicio}
\solucao{
Seja $p(x) = ax^2 + bx + c$ o polinômio de grau $2$ procurado. Pelas informações do enunciado, queremos que $p(-1) = -1, p(0) = 0, p(1) = 1$ e $p(-1) = 1.$ Assim, devemos ter
\[
\begin{cases}
(-1, -1) \Rightarrow a(-1)^2 + b(-1) + c = -1 \\
(0,0) \Rightarrow a \cdot 0^2 + b \cdot 0 + c = 0 \\
(1,1) \Rightarrow a \cdot 1^2 + b \cdot 1 + c = 1 \\
(-1,1) \Rightarrow a \cdot (-1)^2 + b(-1) + c = 1
\end{cases} \Rightarrow \begin{cases}
a - b + c = -1\\
c = 0 \\
a + b + c = 1 \\
a - b + c = 1
\end{cases} \Rightarrow \]\[ \begin{cases}
a - b  = -1\\
a + b = 1 \\
a - b = 1
\end{cases} \Rightarrow \begin{bmatrix}
1 & -1 \\
1 & 1 \\
1 & -1 
\end{bmatrix}\begin{bmatrix}
a \\ 
b
\end{bmatrix} = \begin{bmatrix}
-1 \\
1 \\
1
\end{bmatrix}
\]
Tomando $T = \begin{bmatrix}
1 & -1 \\
1 & 1 \\
1 & -1 
\end{bmatrix},$ $x = (a, b)$ e $b = (-1,1,1),$ temos o sistema $T(x) = b.$ Para encontrar o valor de $x$ que melhor se ajusta à situação, basta resolvermos o sistema $(T^{*}T)(x) = T^{*}(b).$ Temos
\[
T^{*} = \begin{bmatrix}
1 & 1 & 1 \\
-1 & 1 & -1
\end{bmatrix},
\]
daí
\[
(T^{*}T)(x) = T^{*}(b) \Rightarrow \begin{bmatrix}
1 & 1 & 1 \\
-1 & 1 & -1
\end{bmatrix} \begin{bmatrix}
1 & -1 \\
1 & 1 \\
1 & -1 
\end{bmatrix} \begin{bmatrix}
a \\ 
b
\end{bmatrix} = \begin{bmatrix}
1 & 1 & 1 \\
-1 & 1 & -1
\end{bmatrix}\begin{bmatrix}
-1 \\
1 \\
1
\end{bmatrix} \Rightarrow \]\[ \left(\begin{matrix}
3 & -1 \\
-1 & 3
\end{matrix}\right) \begin{bmatrix}
a \\ 
b
\end{bmatrix} = \begin{bmatrix}
1 \\ 
1
\end{bmatrix} \Rightarrow \begin{cases}
3a - b = 1 \\
-a + 3b = 1
\end{cases} \Rightarrow a = b = \frac{1}{2}.
\]
Como já havíamos visto no início que $c = 0,$ concluímos que o polinômio de grau $2$ que melhor se ajusta aos pontos descritos é
\[
p(x) = \frac{1}{2}(x^2 + x).
\]
}


\begin{exercicio}
Sejam $U$ e $V$ espaços com produto interno, $T \in \mathcal{L}(U,V)$ uma aplicação linear de $U$ a $V$ e $T^{*} \in \mathcal{L}(V,U)$ a aplicação adjunta. Considere as afirmações abaixo e assinale quais delas são corretas:
\dividiritens{
\task[\pers{a}] $T$ é sobrejetora se, e somente se, $T^{*}$ é sobrejetora;
\task[\pers{b}] $T$ é injetora se, e somente se, $T^{*}$ é injetora;
\task[\pers{c}] $T$ é injetora se, e somente se, $T^{*}$ é sobrejetora;
\task[\pers{d}] $T$ é bijetora se, e somente se, $T^{*}$ é bijetora.
}
\end{exercicio}
\solucao{Observe que 
\[
\mbox{Im }(T^{*}) = (\Ker T)^{\perp} \quad \mbox{e} \quad (\mbox{Im } T)^{\perp} = \Ker (T^{*})
\]
Suponha então que $T$ seja injetora. Logo, $\mbox{Im }(T^{*}) = \{ 0 \}^{\perp} = V.$ Assim, $T^{*}$ é sobrejetora. A recíproca é análoga. 

Se $T^{*}$ é injetora, então $\Ker T^{*} = \{ 0 \},$ acarretando $\mbox{Im }T = V.$ Assim, $T$ é sobrejetora. A recíproca é análoga. 

Logo, as alternativas corretas são a $c$ e a $d.$
}
\begin{exercicio}
Encontre as formas de Jordan e racional do operador $T \in \mathcal{L}(\mathbb{R}^n)$ se a matriz de $T$ na base canônica do $\mathbb{R}^n$ é a seguinte:
\[
\begin{bmatrix}
1 & 1 & \cdots & 1 \\
2 & 2 & \cdots & 2 \\
\vdots & \vdots & \ddots & \vdots \\
n & n & \cdots & n
\end{bmatrix}
\]
\end{exercicio}
\solucao{
}
\begin{exercicio}
Encontre as formas de Jordan e racional do operador $T \in \mathcal{L}(\mathbb{R}^4)$ e as bases correspondentes se a matriz de $T$ na base canônica do $\mathbb{R}^4$ é a seguinte:
\[
\begin{pmatrix}
-5 & -6 & -15 & 14 \\
4 & 5 & 9 & -6 \\
0 & 0 & 2 & -4 \\
0 & 0 & 1 & -2
\end{pmatrix}
\]
\end{exercicio}
\solucao{
Encontremos os autovalores e autovetores de $T:$
\[
p_T(\lambda) = \det (T - \lambda I) = \det \begin{pmatrix}
-5-\lambda & -6 & -15 & 14 \\
4 & 5-\lambda & 9 & -6 \\
0 & 0 & 2-\lambda & -4 \\
0 & 0 & 1 & -2-\lambda
\end{pmatrix} = \]\[\det \begin{pmatrix}
-5-\lambda & -6  \\
4 & 5-\lambda  
\end{pmatrix} \det \begin{pmatrix}
2-\lambda & -4 \\
1 & -2-\lambda
\end{pmatrix} = (\lambda^2 - 1)\lambda^2 = \lambda^2(\lambda + 1)(\lambda - 1)
\]
Os autovalores são $\lambda_1 = 0,$ com multiplicidade $2,$ $\lambda_2 = 1$ e $\lambda_3 = -1.$ Encontremos agora os autovetores:
\[
v \in \Ker(T - 0I) \Rightarrow \begin{pmatrix}
-5 & -6 & -15 & 14 \\
4 & 5 & 9 & -6 \\
0 & 0 & 2 & -4 \\
0 & 0 & 1 & -2
\end{pmatrix} \begin{pmatrix}
x \\
y \\
z \\
w
\end{pmatrix} = 0 \Rightarrow v = w(-8,4,2,1)
\]
\[
v \in \Ker(T - 1I) \Rightarrow \begin{pmatrix}
-6 & -6 & -15 & 14 \\
4 & 4 & 9 & -6 \\
0 & 0 & 1 & -4 \\
0 & 0 & 1 & -3
\end{pmatrix} \begin{pmatrix}
x \\
y \\
z \\
w
\end{pmatrix} = 0 \Rightarrow v = y(-1,1,0,0)
\]
\[
v \in \Ker(T + 1I) \Rightarrow \begin{pmatrix}
-4 & -6 & -15 & 14 \\
4 & 6 & 9 & -6 \\
0 & 0 & 3 & -4 \\
0 & 0 & 1 & -1
\end{pmatrix} \begin{pmatrix}
x \\
y \\
z \\
w
\end{pmatrix} = 0 \Rightarrow v = y\left(-\frac{3}{2},1,0,0 \right)
\]
Encontremos agora a forma de Jordan e a forma racional.
Para encontrar a forma de Jordan, observe que o autovalor $0$ possui multiplicidade algébrica $2,$ e como $\dim (\Ker(A)) = 1,$ então este corresponde a um bloco de Jordan $2 \times 2.$ Como a multiplicidade algébrica dos outros dois autovalores é $1,$ então esses correspondem a um bloco de Jordan $1 \times 1.$ Assim, a forma de Jordan correspondente é
\[
J = \begin{array}{cc:c:c}
0 & 1 & 0 & 0 \\
0 & 0 & 0 & 0 \\ \hdashline 
0 & 0 & 1 & 0 \\ \hdashline
0 & 0 & 0 & -1
\end{array}
\]
Encontremos a matriz $P$ tal que $A = PJP^{-1}.$
Para isso, observe que precisamos encontrar um autovetor generalizado associado ao autovalor $0.$ Já sabemos que $(-8,4,2,1)$ é um autovalor "genuíno" para $0$. Seja $v = (x,y,z,w)$ o autovetor generalizado procurado. Devemos ter
\[
(T - 0I)v = \begin{pmatrix}
-8 \\
4 \\
2 \\
1
\end{pmatrix} \Rightarrow \begin{pmatrix}
-5 & -6 & -15 & 14 \\
4 & 5 & 9 & -6 \\
0 & 0 & 2 & -4 \\
0 & 0 & 1 & -2
\end{pmatrix} \begin{pmatrix}
x \\
y \\
z \\
w
\end{pmatrix} = \begin{pmatrix}
-8 \\
4 \\
2 \\
1
\end{pmatrix} \Rightarrow \]\[ \begin{cases}
-5x -6y -15z +14w = -8 \\
4x + 5y + 9z -6w = 4 \\
2z -4w = 2 \\
z - 2w = 1
\end{cases} \Rightarrow v = (-5,3,1, 0) + w(-8,4,2,1) \Rightarrow v = (-5,3,1,0)
\]
Logo, $(-5,3,1,0)$ é o autovetor generalizado procurado. Portanto,
\[
P = \begin{pmatrix}
-8 & -5 & -1 & -\frac{3}{2} \\
4 & 3 & 1 & 1 \\
2 & 1 & 0 & 0 \\
1 & 0 & 0 & 0
\end{pmatrix}.
\]
É fácil ver que
\[
PJP^{-1} = \left(\begin{matrix}
-8 & -5 & -1 & \frac{-3}{2} \\
4 & 3 & 1 & 1 \\
2 & 1 & 0 & 0 \\
1 & 0 & 0 & 0
\end{matrix}\right)\left(\begin{matrix}
0 & 1 & 0 & 0 \\
0 & 0 & 0 & 0 \\
0 & 0 & 1 & 0 \\
0 & 0 & 0 & -1
\end{matrix}\right)\left(\begin{matrix}
0 & 0 & 0 & 1 \\
0 & 0 & 1 & -2 \\
2 & 3 & 1 & 2 \\
-2 & -2 & -4 & 0
\end{matrix}\right) = \left(\begin{matrix}
-5 & -6 & -15 & 14 \\
4 & 5 & 9 & -6 \\
0 & 0 & 2 & -4 \\
0 & 0 & 1 & -2
\end{matrix}\right) = T
\]

Vamos agora encontrar a forma racional de $T.$ Para isso, encontremos primeiramente o polinômio minimal de $T.$ Nesse caso, uma rápida verificação mostra que o polinômio minimal coincide com o característico. Daí,
\[
m_T(\lambda) = p_T(\lambda) = \lambda^4 - \lambda^2.
\]
Assim, temos que
\[
\mathbb{R}^4 = Z(v_1, T),
\]
pois o polinômio minimal será o polinômio anulador de $Z(v_1,T),$ de onde $\dim(Z(v_1, T)) = \deg(m_T(\lambda)) = 4.$ Assim, a matriz $R$ corresponde à matriz companheira do polinômio minimal, ou seja,
\[R = \begin{pmatrix}
0 & 0 & 0 & -a_0 \\
1 & 0 & 0 & -a_1 \\
0 & 1 & 0 & -a_2 \\
0 & 0 & 1 &  -a_3
\end{pmatrix} = \begin{pmatrix}
0 & 0 & 0 & 0 \\
1 & 0 & 0 & 0 \\
0 & 1 & 0 & 1 \\
0 & 0 & 1 & 0
\end{pmatrix}
\]
Vamos encontrar agora a matriz $P$ tal que $T = PRP^{-1}.$ Como $\dim(Z(v_1, T)) = 4,$ uma base para $Z(v_1, T)$ é $\{ v_1, T(v_1), T^2(v_1), T^3(v_1) \}.$ Falta agora encontrar o vetor $v_1.$ Como o $T$-anulador de $v_1$ é o polinômio minimal, devemos escolher um vetor de cada um dos autoespaços associados aos respectivos autovalores.\textcolor{Red}{Sejam então $u_1 = (-8,4,2,1), u_2 = (-1,1,0,0)$ e $u_3 = \left( - \frac{3}{2}, 1, 0, 0 \right).$ Daí, tomemos $v_1 = u_1 + u_2 + u_3 = \left(-\frac{21}{2}, 6, 2, 1 \right).$ } Agora, estamos aptos a encontrar uma base para $Z(v_1, T):$
\[
T(v_1) = \begin{pmatrix}
-5 & -6 & -15 & 14 \\
4 & 5 & 9 & -6 \\
0 & 0 & 2 & -4 \\
0 & 0 & 1 & -2
\end{pmatrix} \begin{pmatrix}
1 \\
0 \\
0 \\
0
\end{pmatrix} = \begin{pmatrix}
-4 \\
4 \\
0 \\
0
\end{pmatrix};
\]
\[
T^2(v_1) = \left(\begin{matrix}
1 & 0 & 5 & -2 \\
0 & 1 & -3 & 2 \\
0 & 0 & 0 & 0 \\
0 & 0 & 0 & 0
\end{matrix}\right) \begin{pmatrix}
1 \\
0 \\
0 \\
0
\end{pmatrix} = \begin{pmatrix}
-\frac{5}{2} \\
2 \\
0 \\
0
\end{pmatrix};
\]
\[
T^3(v_1) = \left(\begin{matrix}
-5 & -6 & -7 & -2 \\
4 & 5 & 5 & 2 \\
0 & 0 & 0 & 0 \\
0 & 0 & 0 & 0
\end{matrix}\right) \begin{pmatrix}
-\frac{21}{2} \\
6 \\
2 \\
1
\end{pmatrix} = \begin{pmatrix}
\frac{1}{2} \\
0 \\
0 \\
0
\end{pmatrix}.
\]
Portanto, a base procurada é $\left\{ \left(-\frac{21}{2},6,2,1\right), \left(\frac{1}{2},0,0,0\right), \left(-\frac{5}{2},2,0,0\right), \left(\frac{1}{2},0,0,0\right) \right\},$ e segue que
%\[P = \begin{pmatrix}- \frac{21}{2} & \frac{1}{2} & - \frac{5}{2} & \frac{1}{2} \\6 & 0 & 2 & 0 \\2 & 0 & 0 & 0 \\1 & 0 & 0 & 0\end{pmatrix} \]

}
\begin{exercicio}
Sejam $V$ um $K$-espaço vetorial com produto interno, $T \in \mathcal{L}(V)$ e $W \subseteq V$ um subespaço $T$-invariante. Mostre que se $T$ é normal, então $W$ também é $T^{*}$-invariante.
\end{exercicio}
\solucao{
%https://math.stackexchange.com/questions/436466/if-a-is-normal-then-am-subset-m-rightarrow-am-perp-subset-m-perp
}

\newpage
\section{\textcolor{Floresta}{Prova 3}}


\begin{exercicio}
(3,0) Verifique quais das afirmações abaixo são falsas, justificando cada item adequadamente:
\dividiritens{
\task[\pers{a}] Se $A \in \mathcal{M}_n(\mathbb{C})$ é normal, então $A^t$ é normal;
\task[\pers{b}] Se $A = (a_{ij}) \in \mathcal{M}_n(\mathbb{C})$ é normal, então $\overline{A} = (\overline{a_{ij}})$ é normal;
\task[\pers{c}] Seja $A \in \mathcal{M}_n(\mathbb{C}),$ e considere matrizes $B, C \in \mathcal{M}_n(\mathbb{R}),$ tais que $A = B + iC.$ Se $A$ é normal, então $B$ e $C$ são normais;
\task[\pers{d}] No item anterior, se $B$ e $C$ são normais, então $A = B + iC$ é normal;

\task[\pers{e}] Para qualquer matriz $A \in \mathcal{M}_n(\mathbb{R}),$ a amtriz $A + iA^t \in \mathcal{M}_n(\mathbb{C})$ é normal; 

\task[\pers{f}] Existe um operador normal $T \in \mathcal{L}(\mathbb{R}^3)$ tal que
\[
T(1,2,3) = (1,2,3) \quad \mbox{e} \quad T(1,1,1) = (0,0,0).
\]
}
\end{exercicio}
\solucao{
\dividiritens{
\task[\pers{a}] Se $A$ é normal, então $A^{*}A = AA^{*}.$ Lembrando que $A^{*} = \overline{A}^{t},$ temos que 
\[
(A^t)(A^t)^{*} = A^t(\overline{A}^{t})^t = A^t \overline{A} = (\overline{A}^t A)^t = (\textcolor{Red}{A^{*}A})^t =(\textcolor{Red}{AA^{*}})^t = \]\[ (A^{*})^t A^t = (\overline{A}^t)^t A^t = \overline{A} A^t = (\overline{A}^{t})^t A^t = (A^t)^{*} (A^t)
\]
Portanto, concluímos que $A^t$ é normal.
\task[\pers{b}] Se $A$ é normal, então $A^{*}A = AA^{*}.$ Lembrando que $A^{*} = \overline{A}^{t},$ temos que 
\[
\overline{A}\overline{A}^{*} = \overline{A} \overline{\overline{A}}^t = \overline{A}A^t = (A \overline{A}^t)^t = (\textcolor{Red}{AA^{*}})^t = (\textcolor{Red}{A^{*}A})^t = A^t (A^{*})^t = A^t (\overline{A}^t)^t = A^t \overline{A} = \overline{\overline{A}}^t \overline{A} = \overline{A} ^{*} \overline{A}
\]
Portanto, concluímos que $\overline{A}$ é normal.
\task[\pers{c}]  Considere 
\[
A = \begin{pmatrix}
1 & 1 \\
i & 3 +2i
\end{pmatrix}.
\]
Observe que
\[
A = \begin{pmatrix}
1 & 1 \\
0 & 3
\end{pmatrix} + i \begin{pmatrix}
0 & 0 \\
1 & 2
\end{pmatrix},
\]
mas as matrizes $B = \begin{pmatrix}
1 & 1 \\
0 & 3
\end{pmatrix}$ e $C = \begin{pmatrix}
0 & 0 \\
1 & 2
\end{pmatrix}$ não são normais, já que
\[
BB^{*} = \begin{pmatrix}
1 & 1 \\
0 & 3
\end{pmatrix}\begin{pmatrix}
1 & 0 \\
1 & 3
\end{pmatrix} = \left(\begin{matrix}
2 & 3 \\
3 & 9
\end{matrix}\right) \neq \left(\begin{matrix}
1 & 1 \\
1 & 10
\end{matrix}\right) = \begin{pmatrix}
1 & 0 \\
1 & 3
\end{pmatrix}\begin{pmatrix} 
1 & 1 \\
0 & 3
\end{pmatrix}= B^{*}B
\]
e
\[
CC^{*} = \left(\begin{matrix}
0 & 0 \\
1 & 2
\end{matrix}\right)\left(\begin{matrix}
0 & 1 \\
0 & 2
\end{matrix}\right) = \left(\begin{matrix}
0 & 0 \\
0 & 5
\end{matrix}\right) \neq  \left(\begin{matrix}
1 & 2 \\
2 & 4
\end{matrix}\right) =
\left(\begin{matrix}
0 & 1 \\
0 & 2
\end{matrix}\right) \left(\begin{matrix}
0 & 0 \\
1 & 2
\end{matrix}\right) = C^{*}C
\]
\task[\pers{d}] Sejam $B$ e $C,$ duas matrizes normais com entradas reais, digamos
\[
B = \begin{pmatrix}
1 & 2 \\
2 & 0
\end{pmatrix} \quad \mbox{e} \quad C = \begin{pmatrix}
2 & 0 \\
0 & 5
\end{pmatrix}.
\]
Então,
\[
A = B + iC = \begin{pmatrix}
1+2i & 2 \\
2 & 5i
\end{pmatrix}.
\]
Mas veja que
\[
AA^{*} = \begin{pmatrix}
1+2i & 2 \\
2 & 5i
\end{pmatrix}\begin{pmatrix}
1-2i & 2 \\
2 & -5i
\end{pmatrix} = \left(\begin{matrix}
9 & 2-6i \\
2+6i & 29
\end{matrix}\right) \neq \left(\begin{matrix}
9 & 2+6i \\
2-6i & 29
\end{matrix}\right) = \]\[\begin{pmatrix}
1-2i & 2 \\
2 & -5i
\end{pmatrix}\begin{pmatrix}
1+2i & 2 \\
2 & 5i
\end{pmatrix} = A^{*}A
\]
Assim, $A$ não é normal.

\task[\pers{e}] Vamos mostrar que, se $B, C \in \mathcal{M}_n(\mathbb{R})$ e $BC = CB,$ ou seja, $B$ e $C$ comutam, então $A = B + iC$ é normal.
}

}

\begin{exercicio}
(2,0) Decomponha a matriz
\[
A = \begin{bmatrix}
0 & 2 & 0 \\
1 & 0 & 0 \\
0 & 0 & -1
\end{bmatrix}
\]
num produto de uma matriz positiva $S$ e uma matriz unitária (que nesse caso será ortogonal) $U$.
\end{exercicio}
\solucao{Vamos encontrar os valores singulares e autovetores de $A.$ Temos que
\[
AA^{*} = \begin{bmatrix}
0 & 2 & 0 \\
1 & 0 & 0 \\
0 & 0 & -1
\end{bmatrix}\begin{bmatrix}
0 & 1 & 0 \\
2 & 0 & 0 \\
0 & 0 & -1
\end{bmatrix} = \begin{bmatrix}
4 & 0 & 0 \\
0 & 1 & 0 \\
0 & 0 & 1
\end{bmatrix}
\]
Como $AA^{*}$ já está na forma diagonal, temos diretamente que os autovalores de $AA^{*}$ são $\lambda_1 = 4$ e $\lambda_2 = 1.$ Daí, os valores singulares de $A$ são $r_1 = \sqrt{\lambda_1} = \sqrt{4} = 2$ e $r_2 = \sqrt{\lambda_2} = \sqrt{1} = 1.$

Encontremos agora autovetores de $AA^{*}.$

Seja $v = (x,y,z) \in \Ker(AA^{*} - 4I).$ Então
\[
(AA^{*} - 4I)v = 0 \Rightarrow \begin{bmatrix}
0 & 0 & 0 \\
0 & -3 & 0 \\
0 & 0 & -3
\end{bmatrix} \begin{bmatrix}
x \\
y \\
z
\end{bmatrix} = 0 \Rightarrow v = x(1,0,0)
\]
Logo, $\Ker(AA^{*} - 4I) = \langle (1,0,0) \rangle.$

Seja $v = (x,y,z) \in \Ker(AA^{*} - I).$ Então
\[
(AA^{*} - I)v = 0 \Rightarrow \begin{bmatrix}
3 & 0 & 0 \\
0 & 0 & 0 \\
0 & 0 & 0
\end{bmatrix} \begin{bmatrix}
x \\
y \\
z
\end{bmatrix} = 0 \Rightarrow v = y(0,1,0) + z(0,0,1)
\]
Logo, $\Ker(AA^{*} - 4I) = \langle (0,1,0), (0,0,1) \rangle.$

Assim, $e_1 = (1,0,0), e_2 = (0,1,0)$ e $e_3 = (0, 0, 1)$ são autovetores de $AA^{*}.$ 
Para obter $S$ e $U$, podemos proceder de duas maneiras distintas:

\begin{itemize}
    \item Sabemos que $A(e_i) = r_if_i.$ Com base nisso, é possível encontrar os elementos da base $C = \{f_1, f_2, f_3 \}.$ Realizando os cálculos, vem:
    \[
    \begin{array}{l}
    A(e_1) = r_1f_1 \Rightarrow f_1 = \frac{1}{2} \begin{bmatrix}
0 & 2 & 0 \\
1 & 0 & 0 \\
0 & 0 & -1
\end{bmatrix} \begin{pmatrix}
1 \\
0 \\
0 \\
\end{pmatrix} \Rightarrow f_1 = \left(0, \frac{1}{2}, 0 \right)    \\
              A(e_2) = r_2f_2 \Rightarrow f_2 = \frac{1}{1} \begin{bmatrix}
0 & 2 & 0 \\
1 & 0 & 0 \\
0 & 0 & -1
\end{bmatrix} \begin{pmatrix}
0 \\
1 \\
0 \\
\end{pmatrix} \Rightarrow f_2 = \left(2, 0 , 0 \right)  \\
         A(e_3) = r_3f_3 \Rightarrow f_3 = \frac{1}{1} \begin{bmatrix}
0 & 2 & 0 \\
1 & 0 & 0 \\
0 & 0 & -1
\end{bmatrix} \begin{pmatrix}
0 \\
0 \\
1 \\
\end{pmatrix} \Rightarrow f_3 = \left(0, 0 , -1 \right)  \\
    \end{array}
    \]

Para encontrar $U,$ façamos
\[
\left(\begin{array}{c|c}
e_1 & f_1 \\
e_2 & f_2 \\
e_3 & f_3
\end{array} \right) = \left(\begin{array}{ccc|ccc}
1 & 0 & 0 & 0 & \frac{1}{2} & 0  \\
0 & 1 & 0 & 2 & 0 & 0  \\
0 & 0 & 1 & 0 & 0 & -1
\end{array} \right) \Rightarrow U =  \left(\begin{array}{ccc}
0 & \frac{1}{2} & 0  \\
2 & 0 & 0  \\
0 & 0 & -1
\end{array} \right)^t =  \left(\begin{array}{ccc}
0 & 2 & 0  \\
\frac{1}{2} & 0 & 0  \\
0 & 0 & -1
\end{array} \right)
\]
Calculemos agora a matriz $S.$ Temos que
\[
\left(\begin{array}{c|c}
f_1 & r_1f_1 \\
f_2 & r_2f_2 \\
f_3 & r_3f_3
\end{array} \right) = \left(\begin{array}{ccc|ccc}
0 & \frac{1}{2} & 0 & 0 & 1 & 0 \\
2 & 0 & 0 & 2 & 0 & 0  \\
0 0 & -1 & 0 & 0 & -1
\end{array} \right) \sim \left(\begin{array}{ccc|ccc}
1 & 0 & 0 & 1 & 0 & 0  \\
0 & 1 & 0 & 0 & 2 & 0 \\
0 & 0 & 1 & 0 & 0 & 1
\end{array} \right) \Rightarrow S =  \left(\begin{array}{ccc}
1 & 0 & 0  \\
0 & 2 & 0  \\
0 & 0 & 1
\end{array} \right)
\]
Assim, a matriz positiva procurada é $S = \left(\begin{array}{ccc}
1 & 0 & 0  \\
0 & 2 & 0  \\
0 & 0 & 1
\end{array} \right),$ e a matriz unitária(?) é $U = \left(\begin{array}{ccc}
0 & 2 & 0  \\
\frac{1}{2} & 0 & 0  \\
0 & 0 & -1
\end{array} \right).$ De fato, é fácil ver que
\[
SU =  \left(\begin{array}{ccc}
1 & 0 & 0  \\
0 & 2 & 0  \\
0 & 0 & 1
\end{array} \right) \left(\begin{array}{ccc}
0 & 2 & 0  \\
\frac{1}{2} & 0 & 0  \\
0 & 0 & -1
\end{array} \right) = \left(\begin{array}{ccc}
0 & 2 & 0  \\
1 & 0 & 0  \\
0 & 0 & -1
\end{array} \right) = A.
\]
\item  Como $e_1 = (1,0,0), e_2 = (0,1,0)$ e $e_3 = (0, 0, 1)$ são autovetores de $AA^{*},$ temos que
\[
P = \begin{bmatrix}
e_1 & e_2 & e_3
\end{bmatrix} = \begin{bmatrix}
1 & 0 & 0 \\
0 & 1 & 0 \\
0 & 0 & 1
\end{bmatrix}
\]
Já que $P = I, $ obviamente $P^{*} = I^{*} = I = P.$ Desse modo,
\[
S = P\begin{bmatrix}
r_1 & 0 ¨& 0 \\
0 & r_2 & 0 \\
0 & 0 & r_3
\end{bmatrix}P^{*} =  \begin{bmatrix}
1 & 0 & 0 \\
0 & 1 & 0 \\
0 & 0 & 1
\end{bmatrix} \begin{bmatrix}
2 & 0 & 0 \\
0 & 1 & 0 \\
0 & 0 & 1
\end{bmatrix} \begin{bmatrix}
1 & 0 & 0 \\
0 & 1 & 0 \\
0 & 0 & 1
\end{bmatrix} =  \begin{bmatrix}
2 & 0 & 0 \\
0 & 1 & 0 \\
0 & 0 & 1
\end{bmatrix}
\]
Então, sabemos que $A = SU \Rightarrow U = S^{-1}A.$ Vemos que
\[
S^{-1} = PD^{-1}P^{*} = \begin{bmatrix}
1 & 0 & 0 \\
0 & 1 & 0 \\
0 & 0 & 1
\end{bmatrix} \begin{bmatrix}
\frac{1}{2} & 0 & 0 \\
0 & 1 & 0 \\
0 & 0 & 1
\end{bmatrix} \begin{bmatrix}
1 & 0 & 0 \\
0 & 1 & 0 \\
0 & 0 & 1
\end{bmatrix} =  \begin{bmatrix}
\frac{1}{2} & 0 & 0 \\
0 & 1 & 0 \\
0 & 0 & 1
\end{bmatrix}
\]
Daí,
\[
U = \begin{bmatrix}
\frac{1}{2} & 0 & 0 \\
0 & 1 & 0 \\
0 & 0 & 1
\end{bmatrix} \left(\begin{array}{ccc}
0 & 2 & 0  \\
1 & 0 & 0  \\
0 & 0 & -1
\end{array} \right) =  \left(\begin{array}{ccc}
0 & 1 & 0  \\
1 & 0 & 0  \\
0 & 0 & -1
\end{array} \right)
\]
Logo, $S$ é a matriz positiva procurada e $U$ é a matriz ortogonal.
\end{itemize}
Salientamos que a resposta não é única, o que pode ser claramente visto ao se adotar métodos distintos para encontrar as matrizes $S$ e $U.$
}

\begin{exercicio}
Dada a matriz simétrica positiva
\[
A = \begin{bmatrix}
11 & 7 & 7 \\
7 & 11 & 7 \\
7 & 7 & 11
\end{bmatrix},
\]
ache a raiz quadrada de $A,$ ou seja, encontre uma matriz simétrica positiva $B \in \mathcal{M}_3(\mathbb{R})$ tal que $B^2 = A.$
\end{exercicio}
\solucao{
Observe que como $A$ é uma matriz simétrica, então $A$ é diagonalizável. Assim, existem uma matriz $P$ e uma matriz diagonal $D$ tais que $A = PDP^{-1}.$ Sendo $A$ positiva, então todos os seus autovalores são positivos, o que implica que $B = P \sqrt{D}P^{-1}.$ Vamos encontrar $P$ e $D.$
O polinômio característico de $A$ é
\[
p_A(\lambda) = \det(A - \lambda I) = \det \begin{bmatrix}
11 - \lambda & 7 & 7 \\
7 & 11 - \lambda & 7 \\
7 & 7 & 11- \lambda
\end{bmatrix} = \lambda^3 - 33\lambda^2 + 216 \lambda - 400 = (\lambda - 25)(\lambda - 4)^2
\]
Daí, $\lambda_1 = 25$ e $\lambda_2 = 4$ são os autovalores. Logo, a matriz $D$ é dada por
\[
D = \begin{bmatrix}
25 & 0 & 0 \\
0 & 4 & 0 \\
0 & 0 & 4 
\end{bmatrix}
\]
\[
(A - 25I)v = 0 \Rightarrow \begin{bmatrix}
-14 & 7 & 7 \\
7 & -14 & 7 \\
7 & 7 & -14
\end{bmatrix}\begin{bmatrix}
x \\
y \\
z
\end{bmatrix} \Rightarrow v = x(1,1,1)
\]
Logo, 
\[
\Ker(A - 25I) = \langle (1,1,1) \rangle
\]
Também,
\[
(A - 4I)v = 0 \Rightarrow \begin{bmatrix}
7 & 7 & 7 \\
7 & 7 & 7 \\
7 & 7 & 7
\end{bmatrix}\begin{bmatrix}
x \\
y \\
z
\end{bmatrix} \Rightarrow v = x(-1,1,0) + y(-1,0,1)
\]
Logo, 
\[
\Ker(A - 4I) = \langle (-1,1,0), (-1,0,1) \rangle
\]
Assim, os autovetores são $(1,1,1),(-1,1,0)$ e $(-1,0,1).$ Como a matriz $P$ deve ser ortogonal, precisamos ortogonalizar o conjunto $\{ (-1,1,0),(-1,0,1)\}.$ Pelo processo de ortogonalização de Gram-Schmidt, temos
\[
v_1 =  (-1,1,0), v_2 = (-1,0,1) - \frac{\prin{(-1,0,1),(-1,1,0)}}{\norm{(-1,1,0)}^2}(-1,1,0) = \left(- \frac{1}{2}, - \frac{1}{2}, 1  \right) 
\]
Normalizando os vetores, temos $w_1 = \left( \frac{\sqrt{3}}{3}, \frac{\sqrt{3}}{3}, \frac{\sqrt{3}}{3} \right), w_2 = \left( - \frac{\sqrt{2}}{2}, \frac{\sqrt{2}}{2}, 0 \right)$ e $w_3 = \left( - \frac{\sqrt{6}}{6}, - \frac{\sqrt{6}}{6},  \frac{\sqrt{6}}{3}  \right).$
Logo, 
\[
P = \begin{bmatrix}
 \frac{\sqrt{3}}{3} & - \frac{\sqrt{2}}{2} & - \frac{\sqrt{6}}{6} \\
 \frac{\sqrt{3}}{3} & \frac{\sqrt{2}}{2} & - \frac{\sqrt{6}}{6} \\
 \frac{\sqrt{3}}{3} & 0 &  \frac{\sqrt{6}}{3}  
\end{bmatrix}
\]
Assim, temos que
\[
\sqrt{A} = P\sqrt{D}P^{*} =  \begin{bmatrix}
 \frac{\sqrt{3}}{3} & - \frac{\sqrt{2}}{2} & - \frac{\sqrt{6}}{6} \\
 \frac{\sqrt{3}}{3} & \frac{\sqrt{2}}{2} & - \frac{\sqrt{6}}{6} \\
 \frac{\sqrt{3}}{3} & 0 &  \frac{\sqrt{6}}{3}  
\end{bmatrix}\begin{bmatrix}
\sqrt{25} & 0 & 0 \\
0 & \sqrt{4} & 0 \\
0 & 0 & \sqrt{4} 
\end{bmatrix}\begin{bmatrix}
 \frac{\sqrt{3}}{3} &  \frac{\sqrt{3}}{3}  &  \frac{\sqrt{3}}{3} \\
- \frac{\sqrt{2}}{2} &\frac{\sqrt{2}}{2} & 0 
- \frac{\sqrt{6}}{6} & - \frac{\sqrt{6}}{6}&  \frac{\sqrt{6}}{3}  
\end{bmatrix} = \begin{bmatrix}
 \frac{\sqrt{3}}{3} & - \frac{\sqrt{2}}{2} & - \frac{\sqrt{6}}{6} \\
 \frac{\sqrt{3}}{3} & \frac{\sqrt{2}}{2} & - \frac{\sqrt{6}}{6} \\
 \frac{\sqrt{3}}{3} & 0 &  \frac{\sqrt{6}}{3}  
\end{bmatrix}\begin{bmatrix}
5 & 0 & 0 \\
0 & 2 & 0 \\
0 & 0 & 2
\end{bmatrix}\begin{bmatrix}
 \frac{\sqrt{3}}{3} &  \frac{\sqrt{3}}{3}  &  \frac{\sqrt{3}}{3} \\
- \frac{\sqrt{2}}{2} &\frac{\sqrt{2}}{2} & 0 
- \frac{\sqrt{6}}{6} & - \frac{\sqrt{6}}{6}&  \frac{\sqrt{6}}{3}  
\end{bmatrix} = \begin{bmatrix}
3 & 1 & 1 \\
1 & 3 & 1 \\
1 & 1 & 3
\end{bmatrix}.
\]
Assim, temos que $B = \begin{bmatrix}
3 & 1 & 1 \\
1 & 3 & 1 \\
1 & 1 & 3
\end{bmatrix}.$
}
\begin{exercicio}
(2,0) Prove que um operador $T \in \mathcal{L}(\mathbb{R}^n)$ é diagonalizável se e somente se o espaço $\mathbb{R}^n$ pode ser mundio com um produto interno tal que o operador $T$ seja simétrico relativamente a este produto. Qual é a matriz de Gram que define este produto na base canônica do $\mathbb{R}^n?$
\end{exercicio}
\solucao{}
\begin{exercicio}
(1,0) Fazendo uma troca ortogonal de variáveis conveniente, transforme a forma quadrática
\[
q(x,y,z) = x^2 + y^2 +5z^2 - 6xy - 2xz + 2yz
\]
para uma forma diagonal.
\end{exercicio}
\solucao{
Utilizando o processo de Gauss, dada uma forma quadrática em $\mathbb{R}^n$ $q(x_1, \ldots, x_n) = \sum\limits_{i,j = 1}^n a_{ij}x_ix_j$ com $a_{ij} = a_{ji},$ façamos a seguinte mudança de coordenadas:
\[
\begin{cases}
x_1 = y_1 - \frac{1}{a_{11}}(a_{12}y_2 + \ldots + a_{1n}y_n) \\
x_2 = y_2 \\
\vdots \\
x_n = y_n
\end{cases}
\]
Obteremos assim uma forma quadrática na forma $q(y_1, \ldots, y_n) = a_{11}y_1^2 + q^{\prime}(y_2, \ldots, y_n).$
Apliquemos então o processo de Gauss à forma quadrática do enunciado. 
Observe que a matriz que representa $q(x,y,z)$ é dada por
\[
A = \begin{pmatrix}
1 & -3 & -1 \\
-3 & 1 & 1 \\
-1 & 1 & 5
\end{pmatrix}
\]
de onde vemos que $a_{11} = 1,$ $a_{12} = a_{21} = -3,$ $a_{13} = a_{31} = -1,$ $a_{22} = 1,$ $a_{23} = a_{32} = 1$ e $a_{33} = 5.$
Façamos a seguinte mudança de coordenadas:
\[
\begin{cases}
x = x_1 - \frac{1}{1}(-3y_1 -z_1) \\
y = y_1 \\
z = z_1
\end{cases}
\]
Temos então:
\[
q(x_1,y_1,z_1) = (x_1 + 3y_1 + z_1)^2 + y_1^2 +5z_1^2 - 6(x_1 + 3y_1 + z_1)y_1 - 2(x_1 + 3y_1 + z_1)z_1 + 2y_1z_1 = x_1^2 - 4(y_1z_1 + 2y_1^2 - z_1^2)
\]
Tomemos $q_1(y_1, z_1) = -8y_1^2+4y_1z_1 + 4z_1^2.$ A matriz que representa $q_1(y_1, z_1)$ é dada por
\[
B = \begin{pmatrix}
-8 & 2 \\
 2 & 4
\end{pmatrix}
\]
de onde $b_{11} = -8,$ $b_{12} = b_{21} = 2$ e $b_{22} = 4.$
Façamos a mudança de coordenadas:
\[
\begin{cases}
y_1 = y_2 - \frac{1}{-8}(2 z_2) = y_2 - \frac{z_2}{4} \\
z_1 = z_2
\end{cases}
\]
Temos então:
\[
q(y_2,z_2) =  -8\left( y_2 - \frac{z_2}{4} \right)^2+4\left( y_2 - \frac{z_2}{4} \right)z_1 + 4z_2^2 = -8y_2^2 - \frac{9}{2}z_2^2.
\]
Assim, a forma diagonal de  $q(x,y,z)$ é 
\[
q(x^{\prime}, y^{\prime}, z^{\prime}) = (x^{\prime})^2 - 8 (y^{\prime})^2 - \frac{9}{2}(z^{\prime})^2.
\]





}
 \end{document}
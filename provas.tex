\documentclass[11pt,a4paper]{article}
\usepackage{estilosexercicios}

% ---------------------------------------------------
\title{Álgebra Linear}
\author{MAT5730}
\date{2 semestre de 2019}

\begin{document}
\maketitle
\tableofcontents
\newpage
\begin{comment}

\begin{center}
\large\textbf{\textcolor{Floresta}{Provas}}\\
\end{center}

\end{comment}

\section{\textcolor{Floresta}{Prova 1}}

\begin{exercicio} Sejam $a,b,c,d \in \mathbb{R}.$ Encontre o valor de
\[
\det \begin{bmatrix}
a & b & c & d\\
9 & 8 & 7 & 6\\
1 & 1 & 1 & 1\\
2020 & 2018 & 2017 & 2016
\end{bmatrix}
\]
\end{exercicio}
\solucao{
Como o determinante é uma forma $3$-linear nas linhas da matriz, podemos notar que
\[
\det \begin{bmatrix}
a & b & c & d\\
9 & 8 & 7 & 6\\
1 & 1 & 1 & 1\\
2020 & 2018 & 2017 & 2016
\end{bmatrix} = \det \begin{bmatrix}
a & b & c & d\\
6 & 6 & 6 & 6\\
1 & 1 & 1 & 1\\
2016 & 2016 & 2016 & 2016
\end{bmatrix}
 + \det \begin{bmatrix}
a & b & c & d\\
3 & 2 & 1 & 0\\
1 & 1 & 1 & 1\\
4 & 2 & 1 & 0
\end{bmatrix}
\]
O determinante da primeira matriz é nulo, pois tem três linhas com valores iguais. Já o determinante da segunda matriz pode ser facilmente calculado utilizando o Teorema de Laplace:
\[\det \begin{bmatrix}
a & b & c & d\\
3 & 2 & 1 & 0\\
1 & 1 & 1 & 1\\
4 & 2 & 1 & 0
\end{bmatrix} = (-1)^{1+1} a \cdot \textcolor{Green}{\det \begin{bmatrix}
2 & 1 & 0\\
1 & 1 & 1\\
2 & 1 & 0
\end{bmatrix}} + (-1)^{2+1} b \cdot \textcolor{Blue}{\det
\begin{bmatrix}
3  & 1 & 0\\
1  & 1 & 1\\
4  & 1 & 0
\end{bmatrix}} + \]\[(-1)^{3+1} c \cdot \textcolor{Red}{\det \begin{bmatrix}
3 & 2 & 0\\
1 & 1 & 1\\
4 & 2 & 0
\end{bmatrix}} + (-1)^{4+1} d \cdot \textcolor{Laranja}{\det \begin{bmatrix}
3 & 2 & 1 \\
1 & 1 & 1 \\
4 & 2 & 1 \\
\end{bmatrix}} =  (-1)^{1+1} a \cdot \textcolor{Green}{\det \begin{bmatrix}
2 & 1 & 0\\
1 & 1 & 1\\
2 & 1 & 0
\end{bmatrix}} + (-1)^{2+1} b \cdot \textcolor{Blue}{\det
\begin{bmatrix}
3  & 1 & 0\\
1  & 1 & 1\\
4  & 1 & 0
\end{bmatrix}} + \]\[(-1)^{3+1} c \cdot \textcolor{Red}{\det \begin{bmatrix}
3 & 2 & 0\\
1 & 1 & 1\\
4 & 2 & 0
\end{bmatrix}} + (-1)^{4+1} d \cdot \textcolor{Laranja}{\det \begin{bmatrix}
3 & 2 & 1 \\
1 & 1 & 1 \\
4 & 2 & 1 \\
\end{bmatrix}}
\]
}

\begin{exercicio}
 Seja $V$ um $K$-espaço vetorial e $T \in \mathcal{L}(V).$ Seja $W \subseteq V$ um subespaço $T$-invariante de $V.$
\dividiritens{
\task[\pers{a}] Mostre que, se $T$ é diagonalizável, então a restrição $T \upharpoonleft_W$ é diagonalizável.
\task[\pers{b}] Seja $\mbox{Spec } T = \{ \lambda_1, \ldots, \lambda_n \}$ o conjunto de autovalores de $T,$ onde $\lambda_i \neq \lambda_j$ para $i \neq j.$ Quantos subespaços $T$-invariantes o espaço vetorial $V$ possui?
}
\end{exercicio}
\solucao{}
\begin{exercicio}
Encontre o polinômio característico e o polinômio minimal da matriz
\[
\begin{bmatrix}
1 & -1 & 1 & -1 & 1 & -1 \\
1 & -1 & 1 & -1 & 1 & -1 \\
1 & -1 & 1 & -1 & 1 & -1 \\
1 & -1 & 1 & -1 & 1 & -1 \\
1 & -1 & 1 & -1 & 1 & -1 \\
1 & -1 & 1 & -1 & 1 & -1
\end{bmatrix} \in \mathcal{M}_6(K)
\]
 
\end{exercicio}
\solucao{}
\begin{exercicio}
 Dentre as três matrizes abaixos, quais delas são semelhantes?
\[
A = \begin{pmatrix}
1 & 1 & 1 \\
-1& -1 & -1 \\
1 & 1 & 1
\end{pmatrix}  \quad B = \begin{pmatrix}
1 & 0 & 0 \\
0& 0 & 0 \\
0 & 0 &0
\end{pmatrix} \quad C = \begin{pmatrix}
-1 & -2 & -3 \\
2& 4 & 6 \\
-1& -2 &-3
\end{pmatrix}
\]
\end{exercicio}
\solucao{}
\begin{exercicio} Seja $V = \mathcal{P}_3(\mathbb{R})$ o espaço vetorial formado por todos os polinômios de grau menor ou igual a $3.$ Considere o operador $T \in \mathcal{L}(V)$ dado por
\[
T(f(x)) = (x-1) f^{\prime}(x)
\]
\dividiritens{
\task[\pers{a}] Determine a matriz do operador $T$ em $\mathcal{P}_3(\mathbb{R})$ em relação à base canônica desse espaço.
\task[\pers{b}] Encontre o polinômio característico e o polinômio minimal de $T.$
\task[\pers{c}] $T$ é diagonalizável? Em caso afirmativo, apresente uma base de autovetores para $V.$
}
\end{exercicio}
\solucao{
\dividiritens{
\task[\pers{a}] Considere a base canônica $B = \{1, x, x^2, x^3 \}$ de $V.$ Vamos encontrar a matriz que representa $T$ nessa base. Para tanto, note que
\[
T(1) = (x-1)(1)^{\prime} = 0, \quad T(x) = (x-1)(x)^{\prime} = x-1,
\]
\[
T(x^2) = (x-1)(x^2)^{\prime} = 2x^2 - 2x, \quad T(x^3) = (x-1)(x^3)^{\prime} = 3x^3 - 3x^2.
\]
Logo,
\[
[T]_B = \begin{pmatrix}
0  & 0 & 0 & 0 \\
-1 & 1 & 0 & 0 \\
0 & -2 & 2 & 0 \\
0 & 0 & -3 & 3
\end{pmatrix}
\]


\task[\pers{b}] %Para encontrar os autovalores e autovetores de $T,$ precisamos encontrar a matriz que representa $T$ 
Encontremos os autovalores de $[T]_B:$
\[
p_{[T]_B}(\lambda) = \det([T]_B - \lambda I) = \det \begin{pmatrix}
-\lambda & 0 & 0 & 0 \\
-1 & 1-\lambda & 0 & 0 \\
0 & -2 & 2-\lambda & 0 \\
0 & 0 & -3 & 3-\lambda
\end{pmatrix} = \lambda^4 - 6\lambda^3 + 11 \lambda^2 - 6 \lambda = \lambda (\lambda - 1)(\lambda - 2)( \lambda - 3).
\]
Assim, os autovalores são $0, 1, 2$ e $3.$ Agora, vejamos os autovetores:
\[
v = (a, b, c, d) \in \Ker([T]_B - 0 I) \Rightarrow \begin{pmatrix}
0  & 0 & 0 & 0 \\
-1 & 1 & 0 & 0 \\
0 & -2 & 2 & 0 \\
0 & 0 & -3 & 3
\end{pmatrix}\begin{pmatrix}
a \\
b \\
c \\
d
\end{pmatrix} = 0 \Rightarrow \begin{cases}
a + b = 0 \\
-2b + 2c = 0 \\
-3c + 3d = 0
\end{cases} \Rightarrow v = a(1,-1,-1,-1)
\]
}

}

\newpage
\section{\textcolor{Floresta}{Prova 2}}

\begin{exercicio}
 Suponha que $\mathbb{C}^2$ esteja munido de um produto interno $\prin{,}$ e que a base $\mathcal{B} = \{ (1,1), (1,i) \}$ seja ortonormal em relação a $\prin{,}.$
 \dividiritens{
 \task[\pers{a}] Calcule os seguintes produtos: $\prin{(i,1),(-i,i)}, \prin{(-1,1),(1,1)}$ e $\prin{(1,0),(0,1)}.$
  \task[\pers{b}] Seja $T \in \mathcal{L}(\mathbb{C}^2)$ um operador linear cuja matriz em relação à base canônica de $\mathbb{C}^2$ é
  \[
  [T]_{\mbox{can}} = \begin{pmatrix}
  i+1 & -1 \\
  i & 0
  \end{pmatrix}.
  \]
  
  Determine o operador adjunto $T^{*}$ (relativamente ao produto interno $\prin{,}$ acima) e verifique se $T$ é normal.
 }
\end{exercicio}
\solucao{
\dividiritens{
\task[\pers{a}] Sejam $e_1 = (1,1)$ e $e_2 = (1,i).$ Como $\mathcal{B}$ é ortonormal, então $\prin{e_i, e_j} = \delta_{ij}.$ Com base nisso, vamos calcular cada um dos produtos requisitados:
\begin{itemize}
    \item $\prin{(i,1),(-i,i)}:$
\end{itemize}

}

}

\begin{exercicio}
Determine o polinômio de grau $2$ cujo gráfico melhor se ajusta aos pontos $(-1,-1), (0,0), (1,1)$ e $(-1,1).$
\end{exercicio}
\solucao{



}

\begin{exercicio}
Sejam $U$ e $V$ espaços com produto interno, $T \in \mathcal{L}(U,V)$ uma aplicação linear de $U$ a $V$ e $T^{*} \in \mathcal{L}(V,U)$ a aplicação adjunta. Considere as afirmações abaixo e assinale quais delas são corretas:
\dividiritens{
\task[\pers{a}] $T$ é sobrejetora se, e somente se, $T^{*}$ é sobrejetora;
\task[\pers{b}] $T$ é injetora se, e somente se, $T^{*}$ é injetora;
\task[\pers{c}] $T$ é injetora se, e somente se, $T^{*}$ é sobrejetora;
\task[\pers{d}] $T$ é bijetora se, e somente se, $T^{*}$ é bijetora.
}
\end{exercicio}
\solucao{Observe que 
\[
\mbox{Im }(T^{*}) = (\Ker T)^{\perp} \quad \mbox{e} \quad (\mbox{Im } T)^{\perp} = \Ker (T^{*})
\]
Suponha então que $T$ seja injetora. Logo, $\mbox{Im }(T^{*}) = \{ 0 \}^{\perp} = V.$ Assim, $T^{*}$ é sobrejetora. 

Se $T^{*}$ é injetora, então $\Ker T^{*} = \{ 0 \},$ acarretando $\mbox{Im }T = V.$ Assim, $T$ é injetora. 

Logo, as alternativas corretas são a $T$ e a $T.$
}
\begin{exercicio}
Encontre as formas de Jordan e racional do operador $T \in \mathcal{L}(\mathbb{R}^n)$ se a matriz de $T$ na base canônica do $\mathbb{R}^n$ é a seguinte:
\[
\begin{bmatrix}
1 & 1 & \cdots & 1 \\
2 & 2 & \cdots & 2 \\
\vdots & \vdots & \ddots & \vdots \\
n & n & \cdots & n
\end{bmatrix}
\]
\end{exercicio}
\solucao{
}
\begin{exercicio}
Encontre as formas de Jordan e racional do operador $T \in \mathcal{L}(\mathbb{R}^4)$ e as bases correspondentes se a matriz de $T$ na base canônica do $\mathbb{R}^4$ é a seguinte:
\[
\begin{pmatrix}
-5 & -6 & -15 & 14 \\
4 & 5 & 9 & -6 \\
0 & 0 & 2 & -4 \\
0 & 0 & 1 & -2
\end{pmatrix}
\]
\end{exercicio}
\solucao{}
\begin{exercicio}
Sejam $V$ um $K$-espaço vetorial com produto interno, $T \in \mathcal{L}(V)$ e $W \subseteq V$ um subespaço $T$-invariante. Mostre que se $T$ é normal, então $W$ também é $T^{*}$-invariante.
\end{exercicio}
\solucao{}

\newpage
\section{\textcolor{Floresta}{Prova 2}}


\begin{exercicio}
(3,0) Verifique quais das afirmações abaixo são falsas, justificando cada item adequadamente:
\dividiritens{
\task[\pers{a}] Se $A \in \mathcal{M}_n(\mathbb{C})$ é normal, então $A^t$ é normal;
\task[\pers{b}] Se $A = (a_{ij}) \in \mathcal{M}_n(\mathbb{C})$ é normal, então $\overline{A} = (\overline{a_{ij}})$ é normal;
\task[\pers{c}] Seja $A \in \mathcal{M}_n(\mathbb{C}),$ e considere matrizes $B, C \in \mathcal{M}_n(\mathbb{R}),$ tais que $A = B + iC.$ Se $A$ é normal, então $B$ e $C$ são normais;
\task[\pers{d}] No item anterior, se $B$ e $C$ são normais, então $A = B + iC$ é normal;

\task[\pers{e}] Para qualquer matriz $A \in \mathcal{M}_n(\mathbb{R}),$ a amtriz $A + iA^t \in \mathcal{M}_n(\mathbb{C})$ é normal; 

\task[\pers{f}] Existe um operador normal $T \in \mathcal{L}(\mathbb{R}^3)$ tal que
\[
T(1,2,3) = (1,2,3) \quad \mbox{e} \quad T(1,1,1) = (0,0,0).
\]
}
\end{exercicio}
\solucao{}

\begin{exercicio}
(2,0) Decomponha a matriz
\[
A = \begin{bmatrix}
0 & 2 & 0 \\
1 & 0 & 0 \\
0 & 0 & -1
\end{bmatrix}
\]
num produto de uma matriz positiva $S$ e uma matriz unitária (que nesse caso será ortogonal) $U$.
\end{exercicio}
\solucao{Vamos encontrar os valores singulares e autovetores de $A.$ Temos que
\[
AA^{*} = \begin{bmatrix}
0 & 2 & 0 \\
1 & 0 & 0 \\
0 & 0 & -1
\end{bmatrix}\begin{bmatrix}
0 & 1 & 0 \\
2 & 0 & 0 \\
0 & 0 & -1
\end{bmatrix} = \begin{bmatrix}
4 & 0 & 0 \\
0 & 1 & 0 \\
0 & 0 & 1
\end{bmatrix}
\]
Como $AA^{*}$ já está na forma diagonal, temos diretamente que os autovalores de $AA^{*}$ são $\lambda_1 = 4$ e $\lambda_2 = 1.$ Daí, os valores singulares de $A$ são $r_1 = \sqrt{\lambda_1} = \sqrt{4} = 2$ e $r_2 = \sqrt{\lambda_2} = \sqrt{1} = 1.$

Encontremos agora autovetores de $AA^{*}.$

Seja $v = (x,y,z) \in \Ker(AA^{*} - 4I).$ Então
\[
(AA^{*} - 4I)v = 0 \Rightarrow \begin{bmatrix}
0 & 0 & 0 \\
0 & -3 & 0 \\
0 & 0 & -3
\end{bmatrix} \begin{bmatrix}
x \\
y \\
z
\end{bmatrix} = 0 \Rightarrow v = x(1,0,0)
\]
Logo, $\Ker(AA^{*} - 4I) = \langle (1,0,0) \rangle.$

Seja $v = (x,y,z) \in \Ker(AA^{*} - I).$ Então
\[
(AA^{*} - I)v = 0 \Rightarrow \begin{bmatrix}
3 & 0 & 0 \\
0 & 0 & 0 \\
0 & 0 & 0
\end{bmatrix} \begin{bmatrix}
x \\
y \\
z
\end{bmatrix} = 0 \Rightarrow v = y(0,1,0) + z(0,0,1)
\]
Logo, $\Ker(AA^{*} - 4I) = \langle (0,1,0), (0,0,1) \rangle.$

Assim, $e_1 = (1,0,0), e_2 = (0,1,0)$ e $e_3 = (0, 0, 1)$ são autovetores de $AA^{*}.$ 
Para obter $S$ e $U$, podemos proceder de duas maneiras distintas:

\begin{itemize}
    \item Sabemos que $A(e_i) = r_if_i.$ Com base nisso, é possível encontrar os elementos da base $C = \{f_1, f_2, f_3 \}.$ Realizando os cálculos, vem:
    \[
    \begin{array}{l}
    A(e_1) = r_1f_1 \Rightarrow f_1 = \frac{1}{2} \begin{bmatrix}
0 & 2 & 0 \\
1 & 0 & 0 \\
0 & 0 & -1
\end{bmatrix} \begin{pmatrix}
1 \\
0 \\
0 \\
\end{pmatrix} \Rightarrow f_1 = \left(0, \frac{1}{2}, 0 \right)    \\
              A(e_2) = r_2f_2 \Rightarrow f_2 = \frac{1}{1} \begin{bmatrix}
0 & 2 & 0 \\
1 & 0 & 0 \\
0 & 0 & -1
\end{bmatrix} \begin{pmatrix}
0 \\
1 \\
0 \\
\end{pmatrix} \Rightarrow f_2 = \left(2, 0 , 0 \right)  \\
         A(e_3) = r_3f_3 \Rightarrow f_3 = \frac{1}{1} \begin{bmatrix}
0 & 2 & 0 \\
1 & 0 & 0 \\
0 & 0 & -1
\end{bmatrix} \begin{pmatrix}
0 \\
0 \\
1 \\
\end{pmatrix} \Rightarrow f_3 = \left(0, 0 , -1 \right)  \\
    \end{array}
    \]

Para encontrar $U,$ façamos
\[
\left(\begin{array}{c|c}
e_1 & f_1 \\
e_2 & f_2 \\
e_3 & f_3
\end{array} \right) = \left(\begin{array}{ccc|ccc}
1 & 0 & 0 & 0 & \frac{1}{2} & 0  \\
0 & 1 & 0 & 2 & 0 & 0  \\
0 & 0 & 1 & 0 & 0 & -1
\end{array} \right) \Rightarrow U =  \left(\begin{array}{ccc}
0 & \frac{1}{2} & 0  \\
2 & 0 & 0  \\
0 & 0 & -1
\end{array} \right)^t =  \left(\begin{array}{ccc}
0 & 2 & 0  \\
\frac{1}{2} & 0 & 0  \\
0 & 0 & -1
\end{array} \right)
\]
Calculemos agora a matriz $S.$ Temos que
\[
\left(\begin{array}{c|c}
f_1 & r_1f_1 \\
f_2 & r_2f_2 \\
f_3 & r_3f_3
\end{array} \right) = \left(\begin{array}{ccc|ccc}
0 & \frac{1}{2} & 0 & 0 & 1 & 0 \\
2 & 0 & 0 & 2 & 0 & 0  \\
0 0 & -1 & 0 & 0 & -1
\end{array} \right) \sim \left(\begin{array}{ccc|ccc}
1 & 0 & 0 & 1 & 0 & 0  \\
0 & 1 & 0 & 0 & 2 & 0 \\
0 & 0 & 1 & 0 & 0 & 1
\end{array} \right) \Rightarrow S =  \left(\begin{array}{ccc}
1 & 0 & 0  \\
0 & 2 & 0  \\
0 & 0 & 1
\end{array} \right)
\]
Assim, a matriz positiva procurada é $S = \left(\begin{array}{ccc}
1 & 0 & 0  \\
0 & 2 & 0  \\
0 & 0 & 1
\end{array} \right),$ e a matriz unitária(?) é $U = \left(\begin{array}{ccc}
0 & 2 & 0  \\
\frac{1}{2} & 0 & 0  \\
0 & 0 & -1
\end{array} \right).$ De fato, é fácil ver que
\[
SU =  \left(\begin{array}{ccc}
1 & 0 & 0  \\
0 & 2 & 0  \\
0 & 0 & 1
\end{array} \right) \left(\begin{array}{ccc}
0 & 2 & 0  \\
\frac{1}{2} & 0 & 0  \\
0 & 0 & -1
\end{array} \right) = \left(\begin{array}{ccc}
0 & 2 & 0  \\
1 & 0 & 0  \\
0 & 0 & -1
\end{array} \right) = A.
\]
\item  Como $e_1 = (1,0,0), e_2 = (0,1,0)$ e $e_3 = (0, 0, 1)$ são autovetores de $AA^{*},$ temos que
\[
P = \begin{bmatrix}
e_1 & e_2 & e_3
\end{bmatrix} = \begin{bmatrix}
1 & 0 & 0 \\
0 & 1 & 0 \\
0 & 0 & 1
\end{bmatrix}
\]
Já que $P = I, $ obviamente $P^{*} = I^{*} = I = P.$ Desse modo,
\[
S = P\begin{bmatrix}
r_1 & 0 ¨& 0 \\
0 & r_2 & 0 \\
0 & 0 & r_3
\end{bmatrix}P^{*} =  \begin{bmatrix}
1 & 0 & 0 \\
0 & 1 & 0 \\
0 & 0 & 1
\end{bmatrix} \begin{bmatrix}
2 & 0 & 0 \\
0 & 1 & 0 \\
0 & 0 & 1
\end{bmatrix} \begin{bmatrix}
1 & 0 & 0 \\
0 & 1 & 0 \\
0 & 0 & 1
\end{bmatrix} =  \begin{bmatrix}
2 & 0 & 0 \\
0 & 1 & 0 \\
0 & 0 & 1
\end{bmatrix}
\]
Então, sabemos que $A = SU \Rightarrow U = S^{-1}A.$ Vemos que
\[
S^{-1} = PD^{-1}P^{*} = \begin{bmatrix}
1 & 0 & 0 \\
0 & 1 & 0 \\
0 & 0 & 1
\end{bmatrix} \begin{bmatrix}
\frac{1}{2} & 0 & 0 \\
0 & 1 & 0 \\
0 & 0 & 1
\end{bmatrix} \begin{bmatrix}
1 & 0 & 0 \\
0 & 1 & 0 \\
0 & 0 & 1
\end{bmatrix} =  \begin{bmatrix}
\frac{1}{2} & 0 & 0 \\
0 & 1 & 0 \\
0 & 0 & 1
\end{bmatrix}
\]
Daí,
\[
U = \begin{bmatrix}
\frac{1}{2} & 0 & 0 \\
0 & 1 & 0 \\
0 & 0 & 1
\end{bmatrix} \left(\begin{array}{ccc}
0 & 2 & 0  \\
1 & 0 & 0  \\
0 & 0 & -1
\end{array} \right) =  \left(\begin{array}{ccc}
0 & 1 & 0  \\
1 & 0 & 0  \\
0 & 0 & -1
\end{array} \right)
\]
Logo, $S$ é a matriz positiva procurada e $U$ é a matriz ortogonal.
\end{itemize}
Salientamos que a resposta não é única, o que pode ser claramente visto ao se adotar métodos distintos para encontrar as matrizes $S$ e $U.$
}

\begin{exercicio}
Dada a matriz simétrica positiva
\[
A = \begin{bmatrix}
11 & 7 & 7 \\
7 & 11 & 7 \\
7 & 7 & 11
\end{bmatrix},
\]
ache a raiz quadrada de $A,$ ou seja, encontre uma matriz simétrica positiva $B \in \mathcal{M}_3(\mathbb{R})$ tal que $B^2 = A.$
\end{exercicio}
\solucao{
Observe que como $A$ é uma matriz simétrica, então $A$ é diagonalizável. Assim, existem uma matriz $P$ e uma matriz diagonal $D$ tais que $A = PDP^{-1}.$ Sendo $A$ positiva, então todos os seus autovalores são positivos, o que implica que $B = P \sqrt{D}P^{-1}.$ Vamos encontrar $P$ e $D.$
O polinômio característico de $A$ é
\[
p_A(\lambda) = \det(A - \lambda I) = \det \begin{bmatrix}
11 - \lambda & 7 & 7 \\
7 & 11 - \lambda & 7 \\
7 & 7 & 11- \lambda
\end{bmatrix} = \lambda^3 - 33\lambda^2 + 216 \lambda - 400 = (\lambda - 25)(\lambda - 4)^2
\]
Daí, $\lambda_1 = 25$ e $\lambda_2 = 4$ são os autovalores. 
\[
(A - 25I)v = 0 \Rightarrow \begin{bmatrix}
-14 & 7 & 7 \\
7 & -14 & 7 \\
7 & 7 & -14
\end{bmatrix}\begin{bmatrix}
x \\
y \\
z
\end{bmatrix} \Rightarrow v = x(1,1,1)
\]
Logo, 
\[
\Ker(A - 25I) = \langle (1,1,1) \rangle
\]
Também,
\[
(A - 4I)v = 0 \Rightarrow \begin{bmatrix}
7 & 7 & 7 \\
7 & 7 & 7 \\
7 & 7 & 7
\end{bmatrix}\begin{bmatrix}
x \\
y \\
z
\end{bmatrix} \Rightarrow v = x(-1,1,0) + y(-1,0,1)
\]
Logo, 
\[
\Ker(A - 4I) = \langle (-1,1,0), (-1,0,1) \rangle
\]
Assim, os autovetores
}
\begin{exercicio}
(2,0) Prove que um operador $T \in \mathcal{L}(\mathbb{R}^n)$ é diagonalizável se e somente se o espaço $\mathbb{R}^n$ pode ser mundio com um produto interno tal que o operador $T$ seja simétrico relativamente a este produto. Qual é a matriz de Gram que define este produto na base canônica do $\mathbb{R}^n?$
\end{exercicio}
\solucao{}
\begin{exercicio}
(1,0) Fazendo uma troca ortogonal de variáveis conveniente, transforme a forma quadrática
\[
q(x) = x^2 + y^2 +5z^2 - 6xy - 2xz + 2yz
\]
para uma forma diagonal.
\end{exercicio}
\solucao{}
 \end{document}
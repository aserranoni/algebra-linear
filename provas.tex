\documentclass[11pt,a4paper]{article}
\usepackage{estilosexercicios}

% ---------------------------------------------------
\title{Álgebra Linear}
\author{MAT5730}
\date{2 semestre de 2019}

\begin{document}
\maketitle
\tableofcontents
\newpage
\begin{comment}

\begin{center}
\large\textbf{\textcolor{Floresta}{Provas}}\\
\end{center}

\end{comment}

\section{\textcolor{Floresta}{Prova 1}}

\begin{exercicio} Sejam $a,b,c,d \in \mathbb{R}.$ Encontre o valor de
\[
\det \begin{bmatrix}
a & b & c & d\\
9 & 8 & 7 & 6\\
1 & 1 & 1 & 1\\
2020 & 2018 & 2017 & 2016
\end{bmatrix}
\]
\end{exercicio}
\solucao{
Como o determinante é uma forma $3$-linear nas linhas da matriz, podemos notar que
\[
\det \begin{bmatrix}
a & b & c & d\\
9 & 8 & 7 & 6\\
1 & 1 & 1 & 1\\
2020 & 2018 & 2017 & 2016
\end{bmatrix} = \det \begin{bmatrix}
a & b & c & d\\
6 & 6 & 6 & 6\\
1 & 1 & 1 & 1\\
2016 & 2016 & 2016 & 2016
\end{bmatrix}
 + \det \begin{bmatrix}
a & b & c & d\\
3 & 2 & 1 & 0\\
1 & 1 & 1 & 1\\
4 & 2 & 1 & 0
\end{bmatrix}
\]
O determinante da primeira matriz é nulo, pois tem três linhas com valores iguais. Já o determinante da segunda matriz pode ser facilmente calculado utilizando o Teorema de Laplace:
\[\det \begin{bmatrix}
a & b & c & d\\
3 & 2 & 1 & 0\\
1 & 1 & 1 & 1\\
4 & 2 & 1 & 0
\end{bmatrix} = (-1)^{1+1} a \cdot \textcolor{Green}{\det \begin{bmatrix}
2 & 1 & 0\\
1 & 1 & 1\\
2 & 1 & 0
\end{bmatrix}} + (-1)^{2+1} b \cdot \textcolor{Blue}{\det
\begin{bmatrix}
3  & 1 & 0\\
1  & 1 & 1\\
4  & 1 & 0
\end{bmatrix}} + \]\[(-1)^{3+1} c \cdot \textcolor{Red}{\det \begin{bmatrix}
3 & 2 & 0\\
1 & 1 & 1\\
4 & 2 & 0
\end{bmatrix}} + (-1)^{4+1} d \cdot \textcolor{Laranja}{\det \begin{bmatrix}
3 & 2 & 1 \\
1 & 1 & 1 \\
4 & 2 & 1 \\
\end{bmatrix}} =  (-1)^{1+1} a \cdot \textcolor{Green}{\det \begin{bmatrix}
2 & 1 & 0\\
1 & 1 & 1\\
2 & 1 & 0
\end{bmatrix}} + (-1)^{2+1} b \cdot \textcolor{Blue}{\det
\begin{bmatrix}
3  & 1 & 0\\
1  & 1 & 1\\
4  & 1 & 0
\end{bmatrix}} + \]\[(-1)^{3+1} c \cdot \textcolor{Red}{\det \begin{bmatrix}
3 & 2 & 0\\
1 & 1 & 1\\
4 & 2 & 0
\end{bmatrix}} + (-1)^{4+1} d \cdot \textcolor{Laranja}{\det \begin{bmatrix}
3 & 2 & 1 \\
1 & 1 & 1 \\
4 & 2 & 1 \\
\end{bmatrix}}
\]
}

\begin{exercicio}
 Seja $V$ um $K$-espaço vetorial e $T \in \mathcal{L}(V).$ Seja $W \subseteq V$ um subespaço $T$-invariante de $V.$
\dividiritens{
\task[\pers{a}] Mostre que, se $T$ é diagonalizável, então a restrição $T \upharpoonleft_W$ é diagonalizável.
\task[\pers{b}] Seja $\mbox{Spec } T = \{ \lambda_1, \ldots, \lambda_n \}$ o conjunto de autovalores de $T,$ onde $\lambda_i \neq \lambda_j$ para $i \neq j.$ Quantos subespaços $T$-invariantes o espaço vetorial $V$ possui?
}
\end{exercicio}
\solucao{}
\begin{exercicio}
Encontre o polinômio característico e o polinômio minimal da matriz
\[
\begin{bmatrix}
1 & -1 & 1 & -1 & 1 & -1 \\
1 & -1 & 1 & -1 & 1 & -1 \\
1 & -1 & 1 & -1 & 1 & -1 \\
1 & -1 & 1 & -1 & 1 & -1 \\
1 & -1 & 1 & -1 & 1 & -1 \\
1 & -1 & 1 & -1 & 1 & -1
\end{bmatrix} \in \mathcal{M}_6(K)
\]
 
\end{exercicio}
\solucao{}
\begin{exercicio}
 Dentre as três matrizes abaixos, quais delas são semelhantes?
\[
A = \begin{pmatrix}
1 & 1 & 1 \\
-1& -1 & -1 \\
1 & 1 & 1
\end{pmatrix}  \quad B = \begin{pmatrix}
1 & 0 & 0 \\
0& 0 & 0 \\
0 & 0 &0
\end{pmatrix} \quad C = \begin{pmatrix}
-1 & -2 & -3 \\
2& 4 & 6 \\
-1& -2 &-3
\end{pmatrix}
\]
\end{exercicio}
\solucao{}
\begin{exercicio} Seja $V = \mathcal{P}_3(\mathbb{R})$ o espaço vetorial formado por todos os polinômios de grau menor ou igual a $3.$ Considere o operador $T \in \mathcal{L}(V)$ dado por
\[
T(f(x)) = (x-1) f^{\prime}(x)
\]
\dividiritens{
\task[\pers{a}] Encontre o polinômio característico e o polinômio minimal de $T.$
\task[\pers{b}] $T$ é diagonalizável? Em caso afirmativo, apresente uma base de autovetores para $V.$
}
\end{exercicio}
\solucao{
\dividiritens{
\task[\pers{a}] Considere a base canônica $B = \{1, x, x^2, x^3 \}$ de $V.$ Vamos encontrar a matriz que representa $T$ nessa base. Para tanto, note que
\[
T(1) = (x-1)(1)^{\prime} = 0, \quad T(x) = (x-1)(x)^{\prime} = x-1,
\]
\[
T(x^2) = (x-1)(x^2)^{\prime} = 2x^2 - 2x, \quad T(x^3) = (x-1)(x^3)^{\prime} = 3x^3 - 3x^2.
\]
Logo,
\[
[T]_B = \begin{pmatrix}
0  & 0 & 0 & 0 \\
-1 & 1 & 0 & 0 \\
0 & -2 & 2 & 0 \\
0 & 0 & -3 & 3
\end{pmatrix}
\]


\task[\pers{b}]
}

}

\section{\textcolor{Floresta}{Prova 2}}

\begin{exercicio}
 Suponha que $\mathbb{C}^2$ esteja munido de um produto interno $\prin{,}$ e que a base $\mathcal{B} = \{ (1,1), (1,i) \}$ seja ortonormal em relação a $\prin{,}.$
 \dividiritens{
 \task[\pers{a}] Calcule os seguintes produtos: $\prin{(i,1),(-i,i)}, \prin{(-1,1),(1,1)}$ e $\prin{(1,0),(0,1)}.$
  \task[\pers{b}] Seja $T \in \mathcal{L}(\mathbb{C}^2)$ um operador linear cuja matriz em relação à base canônica de $\mathbb{C}^2$ é
  \[
  [T]_{\mbox{can}} = \begin{pmatrix}
  i+1 & -1 \\
  i & 0
  \end{pmatrix}.
  \]
  
  Determine o operador adjunto $T^{*}$ (relativamente ao produto interno $\prin{,}$ acima) e verifique se $T$ é normal.
 }
\end{exercicio}
\solucao{}

\begin{exercicio}
Determine o polinômio de grau $2$ cujo gráfico melhor se ajusta aos pontos $(-1,-1), (0,0), (1,1)$ e $(-1,1).$
\end{exercicio}
\solucao{



}

\begin{exercicio}
Sejam $U$ e $V$ espaços com produto interno, $T \in \mathcal{L}(U,V)$ uma aplicação linear de $U$ a $V$ e $T^{*} \in \mathcal{L}(V,U)$ a aplicação adjunta. Considere as afirmações abaixo e assinale quais delas são corretas:
\dividiritens{
\task[\pers{a}] $T$ é sobrejetora se, e somente se, $T^{*}$ é sobrejetora;
\task[\pers{b}] $T$ é injetora se, e somente se, $T^{*}$ é injetora;
\task[\pers{c}] $T$ é injetora se, e somente se, $T^{*}$ é sobrejetora;
\task[\pers{d}] $T$ é bijetora se, e somente se, $T^{*}$ é bijetora.
}
\end{exercicio}
\solucao{}
\begin{exercicio}
Encontre as formas de Jordan e racional do operador $T \in \mathcal{L}(\mathbb{R}^n)$ se a matriz de $T$ na base canônica do $\mathbb{R}^n$ é a seguinte:
\[
\begin{bmatrix}
1 & 1 & \cdots & 1 \\
2 & 2 & \cdots & 2 \\
\vdots & \vdots & \ddots & \vdots \\
n & n & \cdots & n
\end{bmatrix}
\]
\end{exercicio}
\solucao{
}
\begin{exercicio}
Encontre as formas de Jordan e racional do operador $T \in \mathcal{L}(\mathbb{R}^4)$ e as bases correspondentes se a matriz de $T$ na base canônica do $\mathbb{R}^4$ é a seguinte:
\[
\begin{pmatrix}
-5 & -6 & -15 & 14 \\
4 & 5 & 9 & -6 \\
0 & 0 & 2 & -4 \\
0 & 0 & 1 & -2
\end{pmatrix}
\]
\end{exercicio}
\solucao{}
\begin{exercicio}
Sejam $V$ um $K$-espaço vetorial com produto interno, $T \in \mathcal{L}(V)$ e $W \subseteq V$ um subespaço $T$-invariante. Mostre que se $T$ é normal, então $W$ também é $T^{*}$-invariante.
\end{exercicio}
\solucao{}
 \end{document}